\documentclass[tikz,border=5pt]{standalone}
\usepackage{amsmath}
\usepackage{pgfplots}
\pgfplotsset{compat=newest}
\usetikzlibrary{fadings}
\usetikzlibrary{shadings}

% Custom arc fading definition
\tikzfading[name=fade out,
    inner color=transparent!0,
    outer color=transparent!100]

% Define the radial shading with a light at the center
\pgfdeclareradialshading{atomshade}{
    \pgfpoint{0cm}{0cm}}
        {
        color(0cm)=(pgftransparent!0);
        color(0.2cm)=(pgftransparent!20);
        color(0.5cm)=(pgftransparent!50);
        color(0.7cm)=(pgftransparent!70);
        color(1cm)=(pgftransparent!100)
    }
    
% Atom definition
\tikzset{atom/.style={circle, shading=atomshade, minimum size=1.0cm}}

\begin{document}
\begin{tikzpicture}

% Atom
\node[atom=gray] (atom1) at (0,0) {};

% Oscillations using arcs with a loop
\foreach \radius in {0.85, 0.95, 1.05, 1.15} {
    \draw[path fading=fade out, thick] (\radius cm - 0.20cm, 0.05cm) arc (15:75:\radius cm);
    \draw[path fading=fade out, thick] (-\radius cm + 0.20cm, -0.05cm) arc (195:255:\radius cm);
}

% Incoming x-ray wave packet
\draw[->,black,variable=\x,samples=200,smooth,domain=-1.7:-1.1]
    plot(\x,{0.1*exp(-(1.45+\x)^2/0.025)*cos(5.0e3*\x)});
\node [above] at (-1.6,0.04) {$\lambda_{\text{in}}$};

% Emitted x-ray packets
\foreach \angle in {95, 135, 165, 275, 315, 350} {
    % Calculate starting points
    \pgfmathsetmacro{\startX}{0.55 * cos(\angle)}
    \pgfmathsetmacro{\startY}{0.55 * sin(\angle)}

    % Draw wave packet with proper rotation
    \draw[->, black, variable=\x, samples=200, smooth, domain=0:0.65]
        plot({
            \startX + \x * cos(\angle) - 0.1 * exp(-(\x-0.325)^2/0.025) * sin(5.0e3*\x) * sin(\angle)
        }, {
            \startY + \x * sin(\angle) + 0.1 * exp(-(\x-0.325)^2/0.025) * sin(5.0e3*\x) * cos(\angle)
        });
}
\node [above] at (-0.5,1.1) {$\lambda_{\text{coh}}$};

\end{tikzpicture}
\end{document}
